%% ----------------------------------------------------------------
%% Thesis.tex -- MAIN FILE (the one that you compile with LaTeX)
%% ---------------------------------------------------------------- 

% Set up the document
\documentclass[a4paper, 11pt, oneside]{Thesis}  % Use the "Thesis" style, based on the ECS Thesis style by Steve Gunn
\graphicspath{Figures/}  % Location of the graphics files (set up for graphics to be in PDF format)

% Include any extra LaTeX packages required
\usepackage[square, numbers, comma, sort&compress]{natbib}  % Use the "Natbib" style for the references in the Bibliography
\usepackage{verbatim}  % Needed for the "comment" environment to make LaTeX comments
\usepackage{vector}  % Allows "\bvec{}" and "\buvec{}" for "blackboard" style bold vectors in maths
\hypersetup{urlcolor=blue, colorlinks=true}  % Colours hyperlinks in blue, but this can be distracting if there are many links.

%% ----------------------------------------------------------------
\begin{document}
\frontmatter      % Begin Roman style (i, ii, iii, iv...) page numbering

% Set up the Title Page
\title  {Structural multi-model coupling with CalculiX and preCICE}
\authors  {\texorpdfstring
            {\href{}{Alexandre Trujillo Boqu\'e}}
            {}
            }
\addresses  {\groupname\\\deptname\\\univname}  % Do not change this here, instead these must be set in the "Thesis.cls" file, please look through it instead
\date       {\today}
\subject    {}
\keywords   {}

\maketitle
%% ----------------------------------------------------------------

\setstretch{1.3}  % It is better to have smaller font and larger line spacing than the other way round

% Define the page headers using the FancyHdr package and set up for one-sided printing
\fancyhead{}  % Clears all page headers and footers
\rhead{\thepage}  % Sets the right side header to show the page number
\lhead{}  % Clears the left side page header

\pagestyle{fancy}  % Finally, use the "fancy" page style to implement the FancyHdr headers

%% ----------------------------------------------------------------
% Declaration Page required for the Thesis, your institution may give you a different text to place here
%Declaration{
%}
%\clearpage  % Declaration ended, now start a new page

%% ----------------------------------------------------------------
% The "Funny Quote Page"
\pagestyle{empty}  % No headers or footers for the following pages

\null\vfill
% Now comes the "Funny Quote", written in italics
\textit{``Write a funny quote here.''}

\begin{flushright}
If the quote is taken from someone, their name goes here
\end{flushright}

\vfill\vfill\vfill\vfill\vfill\vfill\null
\clearpage  % Funny Quote page ended, start a new page
%% ----------------------------------------------------------------

% The Abstract Page
\addtotoc{Abstract}  % Add the "Abstract" page entry to the Contents
\abstract{
\addtocontents{toc}{\vspace{1em}}  % Add a gap in the Contents, for aesthetics

Multi-model coupling of physics simulations is necessary in several contexts. Here the goal is optimization of computational resources\ldots

}

\clearpage  % Abstract ended, start a new page
%% ----------------------------------------------------------------

\setstretch{1.3}  % Reset the line-spacing to 1.3 for body text (if it has changed)

% The Acknowledgements page, for thanking everyone
\acknowledgements{
\addtocontents{toc}{\vspace{1em}}  % Add a gap in the Contents, for aesthetics

The acknowledgements and the people to thank go here, don't forget to include your project advisor\ldots

}
\clearpage  % End of the Acknowledgements
%% ----------------------------------------------------------------

\pagestyle{fancy}  %The page style headers have been "empty" all this time, now use the "fancy" headers as defined before to bring them back


%% ----------------------------------------------------------------
\lhead{\emph{Contents}}  % Set the left side page header to "Contents"
\tableofcontents  % Write out the Table of Contents

%% ----------------------------------------------------------------
\lhead{\emph{List of Figures}}  % Set the left side page header to "List if Figures"
\listoffigures  % Write out the List of Figures

%% ----------------------------------------------------------------
\lhead{\emph{List of Tables}}  % Set the left side page header to "List of Tables"
\listoftables  % Write out the List of Tables

%% ----------------------------------------------------------------
\setstretch{1.5}  % Set the line spacing to 1.5, this makes the following tables easier to read
\clearpage  % Start a new page
\lhead{\emph{Abbreviations}}  % Set the left side page header to "Abbreviations"
\listofsymbols{ll}  % Include a list of Abbreviations (a table of two columns)
{
% \textbf{Acronym} & \textbf{W}hat (it) \textbf{S}tands \textbf{F}or \\
\textbf{LAH} & \textbf{L}ist \textbf{A}bbreviations \textbf{H}ere \\

}

%% ----------------------------------------------------------------
\clearpage  % Start a new page
%\lhead{\emph{Physical Constants}}  % Set the left side page header to "Physical Constants"
%\listofconstants{lrcl}  % Include a list of Physical Constants (a four column table)
%{
% Constant Name & Symbol & = & Constant Value (with units) \\
%Speed of Light & $c$ & $=$ & $2.997\ 924\ 58\times10^{8}\ \mbox{ms}^{-\mbox{s}}$ (exact)\\

%}

%% ----------------------------------------------------------------
\clearpage  %Start a new page
\lhead{\emph{Symbols}}  % Set the left side page header to "Symbols"
\listofnomenclature{lll}  % Include a list of Symbols (a three column table)
{
% symbol & name & unit \\
$a$ & distance & m \\
$P$ & power & W (Js$^{-1}$) \\
& & \\ % Gap to separate the Roman symbols from the Greek
$\omega$ & angular frequency & rads$^{-1}$ \\
}
%% ----------------------------------------------------------------
% End of the pre-able, contents and lists of things
% Begin the Dedication page

\setstretch{1.3}  % Return the line spacing back to 1.3

\pagestyle{empty}  % Page style needs to be empty for this page
\dedicatory{For/Dedicated to/To my\ldots}

\addtocontents{toc}{\vspace{2em}}  % Add a gap in the Contents, for aesthetics


%% ----------------------------------------------------------------
\mainmatter	  % Begin normal, numeric (1,2,3...) page numbering
\pagestyle{fancy}  % Return the page headers back to the "fancy" style

% Include the chapters of the thesis, as separate files
% Just uncomment the lines as you write the chapters

\chapter{Introduction}

% Introduce multi-physics
The field of multi-physics simulations has achieved high popularity amongst scientists and engineers over the last decades [refs multiphysics]. An exponential gain in computational power has introduced the possibility to simulate sub-systems and their interactions simultaneously [refs of relation comput. power-multiphysics], so that they constitute models of complex phenomena, closer to real-life scenarios than single-physics models.
Therefore we have first of all the governing equations for each component of a simulation, and the interactions between them.

% Multi-physics motivated by parallel computing
One of the main sources of computational power and possibilities of scaling is parallel computing. Another reason for the growing interest in multi-physics has been this shift towards parallel computing [refs Benjamin thesis, parallel partitioned, parallel multiphysics etc.]. This is in part related to what has already been mentioned, the increase in computational power, but also to the architecture of a multi-physics setup, where problems in sub-systems can be solved in parallel. In this thesis, parallelism is relevant since we develop a partitioned approach to multi-physics.

%C% MULTIPHYSICS - MULTISCALE - MULTIMODEL

% Monolithic/partitioned concept introduction 
Depending on how we model interactions, we will construct partitioned, monolithic or mixed approaches to multi-physics. %extend...

%Our multi-physics is not really "multi": there is only one physics, structural mechanics
It must be noted that our particular case is a kind of \textit{pseudo}multi-physics, since we only deal with structural physics. Nevertheless, modelling interactions between same-physics sub-systems still represents a challenge, and compared to other multi-physics problems [refs FSI, etc., general multiphysics], there is not much work in structure-structure interaction.
(coupling systems with the same physics can be beneficial in terms of efficiency.)

% Multiphysics translates into using single solver-physics fr each system and then coupling

% Contents of the thesis


\section{Background}

\section{Motivation. Goals.}

Additional motivation, scaling of parallelization running Calculix in a massively parallel setup [refs Benjamin's thesis?].
Fast-paced development, use of exisitng single-physics black-box solvers.



\chapter{Theory}

\section{Basic notions of computational structural meachanics}

Equations of motion, Newton, (see papers you have and their refs)

\section{Partitioned and monolithic approaches}
\subsection{Monolithic structural simulation}
\subsubsection{Gravouil and his friends, all the techniques for mono}
\subsection{Partitioned structural simulations}
\subsubsection{From partitioned FSI}
(Added mass effect here?)
\subsubsection{Actual attempts?}

Always work with matching meshes since it is structure-structure and not fluid-structure, where there [of course]the mesh of the fluid usually does not match. the structure one [refs non-matching meshes] (right?)

\chapter{Software used}

\section{preCICE - A coupling library for partitioned multi-physics simulation}

\section{CalculiX - A three-dimensional structural finite element program}

\chapter{Methodology}
\section{Implementation}

Describe implementation of the structure-structure component in the preCICE adapter for CalculiX.

\section{Tests}
\subsection{Beam test case}
\subsubsection{CalculiX input file/deck}
\subsubsection{Geometry}
\subsection{Comparison partitioned vs. monolithic}
Quantification of error, using L2 norm for displacements and forces, implementation of this.
\subsection{Convergence studies}
\subsubsection{Parameters}
Testing Quasi-Newton postprocessing/convergence acceleration schemes (IQN-ILS) [refs to papaers of QN, see precice papers]. Filters for Quasi-Newton [refs see Benjamin thesis].
Testing parameters limit for filters, convergence measure limit.



\chapter{Results and discussion}

\section{Convergence studies}
Study of error against convergence limit. Error proportional to convergence limit up to an offset (noise, base error,...). Test for several time steps. Error offset decreases with time step (in a reasonble range for time steps) in a very continuous and smooth way. Deduce offset is due to a time splitting error introduced by the coupling [refs. what Benni is doing, precice papers].

\chapter{Conclusions}

%% ----------------------------------------------------------------
% Now begin the Appendices, including them as separate files

\addtocontents{toc}{\vspace{2em}} % Add a gap in the Contents, for aesthetics

\appendix % Cue to tell LaTeX that the following 'chapters' are Appendices

%\input{Appendices/AppendixA}	% Appendix Title


\addtocontents{toc}{\vspace{2em}}  % Add a gap in the Contents, for aesthetics
\backmatter

%% ----------------------------------------------------------------
\label{Bibliography}
\lhead{\emph{Bibliography}}  % Change the left side page header to "Bibliography"
\bibliographystyle{unsrtnat}  % Use the "unsrtnat" BibTeX style for formatting the Bibliography
\bibliography{Bibliography}  % The references (bibliography) information are stored in the file named "Bibliography.bib"

\end{document}  % The End
%% ----------------------------------------------------------------